\documentclass{article}

\usepackage[top=2.54cm, bottom=2.54cm, right=2.54cm, left=2.54cm]{geometry}
\usepackage{graphicx}
\usepackage{xcolor}
\usepackage{hyperref}
\usepackage{caption}
\usepackage{subcaption}
\usepackage{url}
\usepackage{setspace}

\renewcommand{\rmdefault}{phv}
\onehalfspacing

\title{} 
\author{
    Entry Name: \textbf{``LCEE-MC3''}\\
    \textbf{VAST Challenge 2019}\\
    \textbf{\underline{Mini-Challenge 3}}}
\date{}

\begin{document}
%\maketitle

\begin{center}
Entry Name: \textbf{``LCEE-MC3''}\\
\textbf{VAST Challenge 2019}\\
\textbf{\underline{Mini-Challenge 3}}
\end{center}

\noindent
\textbf{Team Members:}\\
Ana Larissa Dias, Federal University of Par\'{a}, \href{mailto:larissa.engcomp@gmail.com}  {\texttt{larissa.engcomp@gmail.com}}   \\
Cassio Batista,   Federal University of Par\'{a}, \href{mailto:cassio.batista.13@gmail.com}{\texttt{cassio.batista.13@gmail.com}} \\
Edwin Rueda,      Federal University of Par\'{a}, \href{mailto:ejrueda95g@gmail.com}       {\texttt{ejrueda95g@gmail.com}}        \\
Erick Campos,     Federal University of Par\'{a}, \href{mailto:erick.c.modesto@gmail.com}  {\texttt{erick.c.modesto@gmail.com}}   \\

\noindent
\textbf{Student Team:} YES \\

\noindent
\textbf{Tools Used:}
\begin{itemize}
    \item \href{https://www.python.org/downloads/}               {Python}           (v3.5.3)
    \item \href{https://bokeh.pydata.org/en/latest/}             {Bokeh}            (v1.2.0)
    \item \href{https://www.nltk.org/}                           {NLTK}             (v3.4.1)
    \item \href{https://textblob.readthedocs.io/}                {TextBlob}         (v0.15.3)
    \item \href{https://www.numpy.org/}                          {NumPy}            (v1.16.0)
    \item \href{https://pandas.pydata.org/}                      {pandas}           (v0.24.2)
    \item \href{https://matplotlib.org/}                         {Matplotlib}       (v3.0.2)
    \item \href{https://github.com/amueller/word_cloud}          {word\_cloud}      (v1.5.0)
    \item \href{https://scikit-learn.org/stable/}                {scikit-learn}     (v0.20.2)
    \item \href{https://nodejs.org/en/download/}                 {Node.js}          (v10.16.0)
    \item \href{https://www.libreoffice.org/}                    {LibreOffice Calc} (v5.2.7.2)
    \item \href{http://man7.org/linux/man-pages/man1/grep.1.html}{grep}             (v2.27)
\end{itemize}

\noindent
\textbf{Approximately how many hours were spent working on this submission in
total?} 114 hours \\

\noindent
\textbf{May we post your submission in the Visual Analytics Benchmark Repository
after VAST Challenge 2019 is complete?} YES \\

\noindent
\textbf{Video:} \url{https://youtube.com/?v=vaietefuder}\\

%\newpage
\noindent
\textbf{Questions} \\
\footnotesize{\textcolor{gray}{The City has been using Y*INT to communicate with
its citizens, even post-earthquake. However, City officials needs additional
information to determine the best way to allocate emergency resources across all
neighborhoods of St. Himark. Your task, using your visual analytics on the
community Y*INT data, is to determine the types of problems that are occurring
across the St.  Himark. Then, advise the City on how to prioritize the
distribution of resources. Keep in mind that not all sources on Y*INT are
reliable, and that priorities may change over time as the state of neighborhoods
also changes.}}

\begin{enumerate}
    \item \textcolor{gray}{Using  visual analytics, characterize conditions
        across the city and recommend how resources should be allocated at 5
        hours and 30 hours after the earthquake. Include evidence from the data
        to support these recommendations. Consider how to allocate resources
        such as road crews, sewer repair crews, power, and rescue teams. 1000
        words, 12 images.}

    In order to discern between reliable and unreliable messages, we have considered
a bar chart of frequency of users that tweeted the most with a hover tool to
show the most frequent words spoken by each user.
Figure~\ref{fig:most_freq_users} shows the unfiltered distribution, in which the
red bars represent users that are probably sellers since their messages include
too many words like ``\emph{deal}'', ``\emph{offer}'', ``\emph{opportunity}'',
``\emph{chances}'', etc. All tweets from those users have therefore been
discarded.

\begin{figure}[!h]
    \centering
    \includegraphics[width=0.95\textwidth]{figs/q1/most_freq_users.png}
    \caption{Most frequent tweeters (usernames). Red bars represent sellers.}
    \label{fig:most_freq_users}
\end{figure}

We have then counted the number of tweets per a small time interval and
presented in a histogram-like plot of Figure~\ref{fig:freq_tweet_overall} that 
shows four peaks, where things possibly might have changed across the city.

\begin{figure}[!h]
    \centering
    \includegraphics[width=0.95\textwidth]{figs/q1/freq_tweet_overall.png}
    \caption{Overall frequency of tweets over time considering all locations.}
    \label{fig:freq_tweet_overall}
\end{figure}

By analysing the heatmap in Figure~\ref{fig:eq_start_heat}, there appears to
have occurred three earthquakes, but the first one seems to have started at 2~PM
of the first day (April 6th). The heatmap is divided by a time interval of one
hour and by neighbourhood location. It was generated by considering a list of
keywords that were tweeted and are similar in meaning (synonyms) and are
directly related to the word ``earthquake''. The list is: 
``\emph{shake}'', ``\emph{shudder}'', ``\emph{vibrate}'', ``\emph{wobble}'',
``\emph{tremor}'', ``\emph{tremble}'', ``\emph{quaver}'', ``\emph{quiver}'',
``\emph{hazard}'', ``\emph{disaster}'', ``\emph{destruction}'', and
``\emph{rubble}''.

\begin{figure}[!h]
    \centering
    \begin{subfigure}[!h]{0.95\textwidth}
        \centering
        \includegraphics[width=1.00\textwidth]{figs/q1/eq_start_heat.png}
        \caption{Heatmap considering cluster of similar keywords (synonyms) for
        the word ``earthquake''.}
        \label{fig:eq_start_heat}
    \vspace{12pt}
    \end{subfigure}
    \begin{subfigure}[!h]{0.95\textwidth}
        \centering
        \includegraphics[width=1.00\textwidth]{figs/q1/eq_start_hbar.png}
        \caption{Bar chart with blue-to-red colormap considering frequency of
        words from 1:00~PM to 3:00~PM of April 6th.}
        \label{fig:eq_start_hbar}
    \end{subfigure}
    \caption{Conditions at the start of the earthquake.}
    \label{fig:eq_start}
\end{figure}

To ensure the heatmap is providing a reliable information, a horizontal bar
chart, shown in Figure~\ref{fig:eq_start_hbar}, was used from 1:30~PM to 
3:00~PM. It counts isolated words and discards words such as adverbs, pronouns,
adjectives, articles and some nouns and verbs that were considered to be useless
such as ``anyone'', ``make'', ``know'', ``food'', ``hate'', etc. Then the Porter
Stemmer from the \texttt{nltk} package was used to clip the words by its
invariant parts (word root), and that root was further reduced to 4-chars only.
A heat-like colormap from blue to red was also included to enhance frequency
distinction. 

The bar chart shows some interesting other words such as ``\emph{feel}'',
``\emph{hear}'', ``\emph{report}'' apart from the keywords aforementioned, in
which ``\emph{earthquake}'' is the most frequent one in accordance with the red
color.

Figure~\ref{fig:map_5h} shows a dynamically-colored SVG map of the city, where
the neighbourhood inner colors range from blue to red in a heatmap fashion
according to the average mean of the colors of the five bigger bars of the
horizontal bar chart of Figure~\ref{fig:eq_start_hbar} for each location. It
appears most reports come from the northwest and southeast of the city.

\begin{figure}[!h]
    \centering
    \includegraphics[width=0.50\textwidth]{figs/q1/cond_5h/cond_5h_svg.png}
    \caption{St. Himark's map 5h after the first earthquake. Lighter shades of
    green represent a higher frequency of messages.}
    \label{fig:map_5h}
\end{figure}

Figure~\ref{fig:eq_cond_5h} shows multiple heatmaps, one per keyword, in a
five-hour time interval from 2:00~PM to 6:59~PM. The charts at the top show
three blank graphs for the keywords ``building'' (Figure~\ref{fig:build_5h}),
``medical'' (Figure~\ref{fig:medical_5h}), and ``road''
(Figure~\ref{fig:road_5h}), which means these resources do not appear to be
requested by any neighbourhood. On the other hand, the 3 graphs at the bottom
show the number of mentions for keywords related to ``sewer and water''
(Figure~\ref{fig:sewer_5h}), ``power'' (Figure~\ref{fig:power_5h}), and ``rain''
(Figure~\ref{fig:rain_5h}).

\begin{figure}[!h]
    \centering
    \begin{subfigure}[!h]{0.32\textwidth}
        \centering
        \includegraphics[width=1.00\textwidth]{figs/q1/cond_5h/cond_5h_build.png}
        \caption{Building}
        \label{fig:build_5h}
    \end{subfigure}
    \begin{subfigure}[!h]{0.32\textwidth}
        \centering
        \includegraphics[width=1.00\textwidth]{figs/q1/cond_5h/cond_5h_medical.png}
        \caption{Medical}
        \label{fig:medical_5h}
    \end{subfigure}
    \begin{subfigure}[!h]{0.32\textwidth}
        \centering
        \includegraphics[width=1.00\textwidth]{figs/q1/cond_5h/cond_5h_road.png}
        \caption{Roads and Bridges}
        \label{fig:road_5h}
    \end{subfigure}
    \begin{subfigure}[!h]{0.32\textwidth}
        \centering
        \includegraphics[width=1.00\textwidth]{figs/q1/cond_5h/cond_5h_sewer.png}
        \caption{Sewer and water}
        \label{fig:sewer_5h}
    \end{subfigure}
    \begin{subfigure}[!h]{0.32\textwidth}
        \centering
        \includegraphics[width=1.00\textwidth]{figs/q1/cond_5h/cond_5h_power.png}
        \caption{Power}
        \label{fig:power_5h}
    \end{subfigure}
    \begin{subfigure}[!h]{0.32\textwidth}
        \centering
        \includegraphics[width=1.00\textwidth]{figs/q1/cond_5h/cond_5h_rain.png}
        \caption{Rain}
        \label{fig:rain_5h}
    \end{subfigure}
    \caption{Conditions after 5h of the first earthquake}
    \label{fig:eq_cond_5h}
\end{figure}

Suggestions for crew allocation is detailed as follows: 

\begin{itemize}
    \item \emph{Sewer and water:} A crew must be sent only to Weston between 
    4:00~PM and 4:59~PM.
    \smallskip 
    \item \emph{Power}: Issues have occurred in Pepper Mill, \textbf{Terrapin
    Springs}, \textbf{Broadview}, Chapparal, \textbf{Southton}, \textbf{Old 
    Town} and Scenic Vista, but we'll consider only the locations 
    highlighted in bold because they have hospitals.
    \begin{itemize}
        \item A crew must be sent to Terrapin Springs between 3:00~PM and
        3:59~PM. Broadview also has a power demand at this time interval but the
        tweet frequency is much lower considering the five-hour period.
        \item Two crews must be sent to Old Town and Southton between 4:00~PM 
        and 4:59~PM.
        \item Lastly, the crew from Terrapin Springs can be reallocated to
        Chapparal between 5:00~PM and 5:59~PM. Although Chapparal does not have
        hospitals, it has been nearly two hours with electrical issues.
    \end{itemize}
    \item \emph{Rescue, sewer and water}: A crew must be sent to Southton only
    because there have been small issues in Weston, Southton and Downtown, and
    therefore Southton is geographically in the middle of such neighbourhoods.
\end{itemize}

Looking at the useful-words colormap over the SVG map of St. Hirmak, shown in
Figure~\ref{fig:map_30h}, it can be
inferred right at the outset that the neighbourhoods that are in most need are
Downtown, Southton, Old Town and Weston.

\begin{figure}[!h]
    \centering
    \includegraphics[width=0.50\textwidth]{figs/q1/cond_30h/cond_30h_svg.png}
    \caption{St. Himark's map 30h after the first earthquake. Yellow and orange 
    shades represent a higher frequency of messages.}
    \label{fig:map_30h}
\end{figure}

\begin{figure}[!h]
    \centering
    \begin{subfigure}[!h]{0.24\textwidth}
        \centering
        \includegraphics[width=1.00\textwidth]{figs/q1/cond_30h/cond_30h_medical.png}
        \caption{Medical}
        \label{fig:medical_30h}
    \end{subfigure}
    \begin{subfigure}[!h]{0.24\textwidth}
        \centering
        \includegraphics[width=1.00\textwidth]{figs/q1/cond_30h/cond_30h_sewer.png}
        \caption{Sewer and water}
        \label{fig:sewer_30h}
    \end{subfigure}
    \begin{subfigure}[!h]{0.24\textwidth}
        \centering
        \includegraphics[width=1.00\textwidth]{figs/q1/cond_30h/cond_30h_rain.png}
        \caption{Rain}
        \label{fig:rain_30h}
    \end{subfigure}
    \begin{subfigure}[!h]{0.24\textwidth}
        \centering
        \includegraphics[width=1.00\textwidth]{figs/q1/cond_30h/cond_30h_road.png}
        \caption{Roads and Bridges}
        \label{fig:roads_30h}
    \end{subfigure}
    \begin{subfigure}[!h]{0.98\textwidth}
        \centering
        \includegraphics[width=1.00\textwidth]{figs/q1/cond_30h/cond_30h_build.png}
        \caption{Building}
        \label{fig:building_30h}
    \end{subfigure}
    \begin{subfigure}[!h]{0.98\textwidth}
        \centering
        \includegraphics[width=1.00\textwidth]{figs/q1/cond_30h/cond_30h_power.png}
        \caption{Power}
        \label{fig:power_30h}
    \end{subfigure}
    \caption{Conditions after 30h of the first earthquake}
    \label{fig:eq_cond_30h}
\end{figure}

By looking at the heatmap of Figure~\ref{fig:eq_cond_30h} for each keyword, it
can be seen that there have been no occurrences for medical, and the ones
related to sewer/water and rain have already been attended within the first five
hours. With respect to roads and bridges in particular, there have been 3
occurrences from 10:00~AM to 11:00~AM of April 8th at Downtown, but this
neighbourhood is under resurfacing maintenance, which implies a road crew is
already working there.

Suggestions for crew allocation is detailed as follows:

\begin{itemize}
    \item \emph{Building}: On April 7th from 7:00~PM to 7:59~PM there have been
    multiple casualties on almost all locations so we would prioritize the
    dark-red-colored ones according to the heatmap of
    Figure~\ref{fig:building_30h}:
    Northwest, Southton, Downtown, and Weston. All four have a high 
    density of buildings and people, apart from being geographically close to
    each other, which can be an advantage for an eventual reallocation of crews
    in the following hours. Terrapin Springs and Cheddarford also have some
    less-intense occurrences, but they must be ignored due to the absence of
    high buildings.
    \item \emph{Power}: According to Figure~\ref{fig:power_30h} on April 7th 
    there have been sporadic, less-intense
    occurrences that could be solved by sending small units to individual
    locations, but from 8:00~AM to 9:00~AM an energy disaster appear to have
    affected almost all neighbourhoods. Again we would prioritize regions where
    the keywords were mentioned the most: Southton, Old Town, and Weston. The
    other can be later attended in the following hours.
\end{itemize}
%%% EOF %%%

    \newpage

    \item \textcolor{gray}{Identify at least 3 times when conditions change in
        a way that warrants a re-allocation of city resources. What were the
        conditions before and after the inflection point? What locations were
        affected? Which resources are involved? Limit your response to 1000
        words and 10 images.}

    As already mentioned, there have been 3 independent earthquakes, as could be 
seen in Figure~\ref{fig:eq_start_heat}: Apr 6th 2:00~PM, Apr 8th 7:00~AM and Apr 
9th 3:00~PM, approximately. Those two last earthquakes are emphasized in
Figure~\ref{fig:eq_2_3_heat}.

\begin{figure}[!h]
    \centering
    \includegraphics[width=1.00\textwidth]{figs/q2/eq_2_3_heat}
    \caption{Heatmap for the two last earquakes using keywords related to the
    word ``earthquake''.}
    \label{fig:eq_2_3_heat}
\end{figure}

\begin{figure}[!h]
    \centering
    \begin{subfigure}[!h]{0.96\textwidth}
        \centering
        \includegraphics[width=1.00\textwidth]{figs/q2/medical_2_3_heat.png}
        \caption{Medical}
        \label{fig:medical_2_3_heat}
    \end{subfigure}
    \begin{subfigure}[!h]{0.96\textwidth}
        \centering
        \includegraphics[width=1.00\textwidth]{figs/q2/sewer_2_3_heat.png}
        \caption{Sewer and water}
        \label{fig:sewer_2_3_heat}
    \end{subfigure}
    \begin{subfigure}[!h]{0.96\textwidth}
        \centering
        \includegraphics[width=1.00\textwidth]{figs/q2/rain_2_3_heat.png}
        \caption{Rain}
        \label{fig:rain_2_3_heat}
    \end{subfigure}
    \begin{subfigure}[!h]{0.96\textwidth}
        \centering
        \includegraphics[width=1.00\textwidth]{figs/q2/road_2_3_heat.png}
        \caption{Roads and Bridges}
        \label{fig:roads_2_3_heat}
    \end{subfigure}
    \begin{subfigure}[!h]{0.96\textwidth}
        \centering
        \includegraphics[width=1.00\textwidth]{figs/q2/build_2_3_heat.png}
        \caption{Building}
        \label{fig:building_2_3_heat}
    \end{subfigure}
    \begin{subfigure}[!h]{0.96\textwidth}
        \centering
        \includegraphics[width=1.00\textwidth]{figs/q2/power_2_3_heat.png}
        \caption{Power}
        \label{fig:power_2_3_heat}
    \end{subfigure}
    \caption{Heatmap of conditions for the two last earthquakes}
    \label{fig:eq_cond_2_3_heat}
\end{figure}

By looking at the heatmap of Figure~\ref{fig:eq_cond_2_3_heat} for each
keyword, it can be seen that the damaged inflicted by the
second earthquake was greater than both first and second ones. It is 

Suggestions for re-allocation of city resources are detailed as follows:   

\begin{itemize}                                                                  
     \item \emph{Medical:} At least 12 hours before 9:00 AM of April 8th, there was no ocurrances regarding
medical keywords. 
\begin{itemize} 
\item On April 8th from 9:00 PM to 10:59 PM, the second earthquake
time, there have been casualties on multiple locations
but we would prioritize the darker colored ones according to the heatmap of Figure 8a:
Northwest, Southton and Downtown. First, a rescue crew must be sent to Northwest
and Southton between     
     9:00~AM and 9:59~AM. Then, from 9:00~AM to 9:59~AM, for being geographically close a crew must be realocated to
Downtown. 

\item Again, between 12:00 PM and 13:00 PM there have been no medical ocurrances but from 13:00 PM
to 16:00 PM, the number of ocurrances increased significantly and affected
almost all neighbourhoods except Safetown and Terrapin Springs. Accordingly, we
would prioritize the dark-red-colored ones, Downtown and Palace Hills, sending
rescue team from 14:00~PM to 15:00~PM for both neighbourhoods.  

\item On following hours after that peak in ocurrances, there have been only sporadic
requests that could be solved by sending small units to individual locations.
 \end{itemize} 
\item \emph{Sewer and water:} According to the Figure~\ref{fig:sewer_2_3_heat}
before 13:00 PM on April 8th there have been less-intense occurrences that could be solved
by sending small units to individual locations. 
 \begin{itemize}                                                                  
 \item From 13:00 PM to 14:00 PM there have been water and sewer requests from
all neighbourhoods, but Oak Willow and Easton. 
Since there was an ocurrance overload we would again prioritize the darker colored
areas. Thus, a water and sewer crew must be sent to Weston and Downtown between 13:00
PM and 14:00 PM. 
\item On following hours the conditions changed back to infrequent ocurrances.
\end{itemize}                                                           
     \smallskip                                                                   
     \item \emph{Rain}: Major issues have occurred in Downton and Weston
according to the coloration of those neighbourhoods shown in Figure~\ref{fig:rain_2_3_heat}                             
     \begin{itemize}                                                              
         \item Between 19:00~PM and 19:59~PM a rescue crew must be sent to
Downtown and Weston. Northwest also has a rain demand at this time interval but the 
 frequency of the request is much lower than the aforementioned neighbourhoods.                                                             
         \item After this time interval there is a conditions change and so between
12:00 PM and 13:00 PM on April 9th the resources can be relocated from the already attended neighbourhoods to the
remaining, such as Northwest, Southton and Broadview.          
     \end{itemize}                                                                
     \item \emph{Roads and Bridges}: 
\item \emph{Building}:
\item \emph{Power}: According to the Figure~\ref{fig:power_2_3_heat} there have
been at least three times that the conditions changed.
begin{itemize}                                                              
          \item Between 07:00~AM and 14:00~PM th                                                                 
          \item                             
      \end{itemize}          
 \end{itemize}   

\begin{figure}[!h]
    \centering
    \begin{subfigure}[!h]{0.98\textwidth}
        \centering
        \includegraphics[width=1.00\textwidth]{figs/q2/eq_2_hbar.png}
        \caption{Bar chart for the second earthquake in a 18h time interval from
        Apr 7th at 6:00~PM to Apr 8th at 12:00~AM.}
        \label{fig:eq_2_hbar}
    \end{subfigure}
    \begin{subfigure}[!h]{0.98\textwidth}
        \centering
        \includegraphics[width=1.00\textwidth]{figs/q2/eq_3_hbar.png}
        \caption{Bar chart for the third earthquake in a 18h time interval from
        Apr 9th at 12:00~PM to Apr 10th at 6:00~AM.}
        \label{fig:eq_3_hbar}
    \end{subfigure}
    \caption{eae}
    \label{fig:eq_2_3_hbar}
\end{figure}

\begin{figure}[!h]
    \centering
    \begin{subfigure}[!h]{0.46\textwidth}
        \centering
        \includegraphics[width=1.00\textwidth]{figs/q2/eq_2_svg.png}
        \caption{Map for the second earthquake. Yellow and orange shades show
        locations where the frequency of tweets is higher. Southton, Downtown,
        Weston and Old Town appear to be, in that order, the neighbourhoods that 
        most need help.}
        \label{fig:eq_2_svg}
    \end{subfigure}
    \hspace{0.75cm}
    \begin{subfigure}[!h]{0.46\textwidth}
        \centering
        \includegraphics[width=1.00\textwidth]{figs/q2/eq_3_svg.png}
        \caption{Map for the third earthquake. Yellow and orange shades show
        locations where the frequency of tweets is higher. Downtown, Weston,
        Pallace Hills, Southwest, and Southton appear to be, in that order, the 
        neighbourhoods that most need help.}
        \label{fig:eq_3_svg}
    \end{subfigure}
    \caption{St. Himark's SVG maps with blue-to-red colormap for the second and
    third earthquakes.}
    \label{fig:eq_2_3_svg}
\end{figure}



    \item \textcolor{gray}{Take the pulse of the community.  How has the
        earthquake affected life in St. Himark? What is the community
        experiencing outside the realm of the first two questions? Show decision
        makers summary information and relevant/characteristic examples. Limit
        your response to 800 words and 8 images.}

    The earthquake has produced some already expected situations such as chaotic
traffic, despair among the population, etc., which has also generated some kind
of uncertainty at the data. For example, the keyword ``\emph{bridge}'' has been
mentioned multiple times, which suggests a structural problem that could've
ended in collapse if analysing the isolated word. However, by analysing the
whole context by means of a bar chart of retweets frequency, as shown in
Figure~\ref{fig:re_bridge_vbar}, we can see that most mentions refer to closing
bridges instead of reporting physical problems at the bridge.

\begin{figure}[!h]
    \centering
    \includegraphics[width=0.95\textwidth]{figs/q3/re_bridge_vbar.png}
    \caption{Bridge closed}
    \label{fig:re_bridge_vbar}
\end{figure}

Another point is with respect to fake news. At 8:00~AM of Apr 8th a tweet
appeared to alarm users about an eventual tsunami, and a lot of retweets have
followed to spread the ``news''. However, as reported on Apr 9th at 9:00~AM, the
city's unique geography prevents such disaster to happen, and according to the
greater number of retweets of this second information we can infer this is the
real truth about tsunamis in St. Himark.

\begin{figure}[!h]
    \centering
    \begin{subfigure}[!h]{0.47\textwidth}
        \centering
        \includegraphics[width=1.00\textwidth]{figs/q3/re_tsunami_fake.png}
        \caption{Tsunami fake}
        \label{fig:re_tsunami_fake}
    \end{subfigure}
    \begin{subfigure}[!h]{0.47\textwidth}
        \centering
        \includegraphics[width=1.00\textwidth]{figs/q3/re_tsunami_truth.png}
        \caption{Tsunami truth}
        \label{fig:re_tsunami_truth}
    \end{subfigure}
    \caption{Tsunami fake news}
    \label{fig:re_tsunami}
\end{figure}

A third discovery made through the retweets frequency bar chart is that there
was a circus show going on in town at the same week, and that after the disaster
that endend with buildings collapsing, the elephants started to be used to help
lifting heavy blocks of rocks in an attempt to search for people or corpses.

\begin{figure}[!h]
    \centering
    \includegraphics[width=0.95\textwidth]{figs/q3/re_circus.png}
    \caption{Circus elephants}
    \label{fig:re_circus}
\end{figure}



    \item \textcolor{gray}{The data for this challenge can be analyzed either as
        a static collection or as a dynamic stream of data, as it would occur in
        a real emergency. Describe how you analyzed the data --- as a static
        collection or a stream. How do you think this choice affected your
        analysis? Limit your response to 200 words and 3 images.}

    The data were analyzed as static collection in order to facilitate
interpretation of the problems. Using a stream analysis many decision-making
for the allocation of resources might be misleading.
Figure~\ref{fig:rain_4_heat} shows an example for the keyword \emph{``rain''}.

\begin{figure}[!h]
    \centering
    \includegraphics[width=1.00\textwidth]{figs/q4/rain_4_heat}
    \caption{Heatmap for the ``\emph{rain}''.}
    \label{fig:rain_4_heat}
\end{figure}

At 6:00 PM, five neighborhoods tweeted about rain or flood. If for each locality
a rescue team was sent, at 7:00 PM where almost all localities reported rain or
flood occurrences, so it would have been more difficult to efficiently serve all
those affected neighborhoods.

\end{enumerate}
\end{document}
